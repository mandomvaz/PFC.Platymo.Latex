\chapter{Introducción}

Al ritmo al que avanza la tecnología, es innegable el hecho de que en un futuro, mas o menos lejano, la práctica totalidad de las viviendas incorpore algún tipo de instalación domótica. 

Pero, ¿qué es la domótica?, según la RAE se define como:
\begin{quote}
    
    (Del lat. domus, casa, e informática).
    
    1. f. Conjunto de sistemas que automatizan las diferentes instalaciones de una vivienda.
    
\end{quote}
De una forma más técnica la podemos definir como el conjunto de procedimientos y tecnologías aplicadas al control y  automatización inteligente del hogar, que permite una gestión eficiente de la energía además de aportar seguridad y confort. 

Aunque actualmente ya existen sistemas completos capaces de manejar sin problemas una vivienda en su totalidad, debido al coste de estas instalaciones, la domótica todavía no se ha popularizado. Estos suelen utilizar buses de transmisión de información que posibilitan una domótica robusta como son el EIB, X10, CEBus o LonWorks/LongTalk. 

Estos sistemas, para poder conseguir las propiedades comentadas anteriormente ,es necesario que recojan  la información de su entorno con sensores y dispongan de la lógica para actuar en consecuencia utilizando actuadores. En definitiva, son una red de sensores y actuadores.

Este proyecto nace para acercar la domótica al gran público, que se dejen de ver estos sistemas como algo de lujo y empiece a ser considerado algo cotidiano. 

Para ello, aunaremos la tecnología de las redes de sensores con los conceptos de la domótica, usando elementos y componentes al alcande de todos, como las placas Arduino o la Raspberry Pi.

 











