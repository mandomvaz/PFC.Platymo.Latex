%numeracion romana
 \pagenumbering{Roman}
 
 \chapter*{Resumen}
 \addcontentsline{toc}{chapter}{Resumen}
 
 La presente memoria describe los conocimientos básicos para entender qué es y cómo funciona un sistema domótico y  utilizando el hardware libre de Arduino  y Raspberry Pi, así como los conceptos de redes de sensores, se puede crear un sistema estable con un presupuesto muy inferior al de las viviendas de alta categoría.
 
 La memoria se divide en 6 capítulos, en los que veremos, a modo de recorrido, desde la situación actual del mercado, a la creación paso por paso de nuestro propio sistema.
 
 En la introducción se define qué es un sistema domótico,  se muestra la problemática actual y nuestra forma de abordar el problema.
 
 Los dos capítulos siguientes describen el estado del arte tanto de la domótica como de las redes de sensores. 
 
 Luego pasamos a ver el hardware y software que se usará en el proyecto.
 
 Ya en el último capítulo, se establecen de manera más formal los objetivos a cumplir, y se describe paso a paso la implementación del sistema.