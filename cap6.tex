\chapter{Implementación}
 A lo largo de este capítulo veremos  la arquitectura del sistema, tanto fisica como lógica, los elementos que lo componen y ....... 

Para ello lo primero es definir el escenario sobre el que implantaremos el sistema.

\section{Escenario}
Para el desarrollo del proyecto necesitamos partir de una base, un escenario concreto sobre el que construir el sistema. Este debe ser lo suficientemente amplio para que una vez acabado podamos extrapolar el sistema domótico a cualquier caso en la vida real.

Para ello, el escenario elegido es una vivienda unifamiliar ya construida y habitada. Ya que así abarcamos a una gran masa de publico que podria beneficiarse de la domótica, a la vez que es facilmente adaptable a viviendas en fase de proyecto.

Una vez elegido el lugar físico en el que se implantará, hay que establecer las funciones que desempeñará el sistema. 

Como vimos anteriormente, un sistema domótico es capaz de manejar muchos aspectos de la vida cotidiana de un hogar, éstas pueden ser alarmas de seguridad, incendios o escape de gases; automatización de persianas y toldos, control de luminarias, apertura y cierre de puertas, y un largo etcétera. 

Actualmente, todas estas funciones son controlables a través de microcontroladores de forma aislada, así que, uno de los objetivos será crear un marco de trabajo en el que poder integrar todas estas funciones trabajando al unísono.

A causa de las limitaciones en tiempo y presupuesto, restringiremos las funciones del sistema al control de las luminarias y                     distintos elementos controlables con un interruptor convencional. 

El escenario quedaría constituido de la siguiente forma:
\begin{description}
    \item[Lugar físico] Vivienda unifamiliar construida y habitada.
        \begin{itemize}
            \item Cableado de las paredes ya existente.
            \item Mezcla heterogénea de luminarias(bombillas de bajo consumo, halógenas, fluorescentes, etc...).
        \end{itemize}
    \item[Funciones del sistema] Limitado a las luminarias.
        \begin{itemize}
            \item Control On/Off.
            \item Deteccion de movimiento en zonas de paso y baños.
            \item Control de Escenas.
            \item Apagado General.
            \item Simulacion de Presencia.
        \end{itemize}
\end{description}

Una vez situados, pasaremos a describir la implementacion del sistema. Lo abordaremos desde dos puntos de vista, a nivel físico, microcontroladores, circuitos y demas elementos tangibles y sus configuraciones; y a nivel lógico, procesos y aplicaciones.

 

\section{Arquitectura física}
A nivel físico el sistema esta basado en una red inalámbrica de sensores convencional, formada por placas Arduino UNO como las motas y una Raspberry Pi como estación base. La Raspberry Pi a su vez, está conectada al router domestico. Este router hace de puente entre la Raspberry Pi y los dispositivos finales que interactuan con el sistema, teléfonos inteligentes, tabletas u ordenadores.  

\subsection{Nodos}
\subsubsection{Ubicación}
Para aprovechar los conductos del cableado eléctrico existente, hemos optado por ubicar un nodo de la red en cada habitacion de la vivienda. De esta manera, introducir los cables necesarios es mas sencillo, puesto que todo queda dentro de cada habitacion. Además, hace al sistema mas modulable al poder incorporar las habitaciones poco a poco.

La red de sensores esta configurada en arquitectura de  malla para asegurar la interconexión de todos los nodos. Ya que en una vivienda, al haber numerosas paredes y obstáculos, con otras configuraciones podrían quedar zonas oscuras en la conexion a la red.

\subsubsection{Equipamiento}
Cada nodo de la red esta equipado con:
\begin{itemize}
    \item Arduino UNO.
    \item Shield XBee.
    \item Sensor de temperatura.
\end{itemize}
Además, los nodos correspondientes a pasillos y baños disponen también de un sensor de presencia.

\subsubsection{Cableado y conexiones}
 Para poder permitir que los nodos puedan interactuar con la vivienda es necesario introducir algunos cambios a esta.
 
 El primer cambio es convertir los interruptores que haya instalados en sensores para el nodo. Esto es tan sencillo como llevar dos cables desde el nodo al interruptor, uno conectado a tierra y el otro en cualquiera de las entradas digitales del arduino. En la figura ??? se puede ver el esquema de la conexión.
 
 FIGURA ESQUEMA CONEXION INTERRUPTOR-ARDUINO
 
 
 El segundo cambio a introducir es posibilitar el accionado de las luminarias de manera digital, es decir, controlado desde el nodo. Para esto utilizaremos el circuito de la figura ????. 
 
 FIGURA CIRCUITO INTERRUPTOR
 
 Este circuito, conocido como relé de estado sólido, nos permite accionar cualquier carga eléctrica desde el nodo. Su funcionamiento es muy sencillo: el nodo activa el led dentro del optoacoplador, este hace que el triac del optoacoplador empiece a conducir, lo cual activa el triac principal, que es el que maneja realmente la carga. 
 
 Hemos elegido como optoacoplador el MOC3041 porque incorpora <<detección de paso por cero>>. El optoacoplador espera a que la señal de  alterna pase por cero para abrir o cerrar el paso de corriente. Este comportamiento reduce el estrés en los aparatos, ya que la señal les llega gradualmente.
  
En la figura ???? se muestra esquema de conexión entre el Nodo, el relé y una lampara.

FIGURA CONEXION ARDUINO-RELE-LAMPARA

Cada Arduino dispone de 14 pines de entrada/salida digitales, estos son los que utilizaremos para conectar los interruptores y actuadores(relés). Los pines analogicos los reservaremos para conectar los sensores, en nuestro caso el sensor de temperatura y el de presencia.


\subsection{Estación base}
La estación base, una  Raspberry Pi en nuestro sistema, es la encargada de gestionar todo el sistema. Va equipada con un modulo XBee, conectado a traves de un puerto USB, para las comunicaciones con la red de sensores. Y para las comunicaciones con el router hay dos opciones: puerto Ethernet, incorporado, directo al router; o añadir un dongle Wi-Fi en el otro puerto USB.








\section{Nivel lógico}

