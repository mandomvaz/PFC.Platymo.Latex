\chapter{Conclusiones y líneas futuras}


A modo de conclusión, en este último capitulo repasaremos  los logros obtenidos, y también veremos de qué forma se podría mejorar o ampliar el sistema.


\section{Conclusiones}

Una vez terminado el proyecto las conclusiones son muy positivas. Se marcaron dos objetivos muy concretos: crear un marco de trabajo unificado para Arduino y Raspberry, e implementar nuestro propio sistema para el control de luminaria. Ambos objetivos se han cumplido ampliamente.

 Hemos creado un sistema de comunicaciones entre Arduino y Raspberry versátil, ampliable, funcional e inalámbrico. Ello queda patente en la consecución del segundo objetivo, nuestro propio sistema. 

Nuestro sistema, como veremos en el siguiente apartado, todavía tiene mucho margen de mejora; pero aun así,  demuestra que es perfectamente posible y viable crear un sistema domótico completo usando elementos sencillos como Arduino y Raspberry Pi.

Para el gran objetivo del proyecto, conseguir un acercamiento del gran público a la domótica, solo el tiempo nos dirá si lo hemos conseguido o no. No obstante, creemos que hemos dado los pasos en la dirección adecuada al concebir un sistema de bajo coste y de fácil instalación y uso.
\clearpage

\section{Líneas futuras}

En cuanto al futuro, las primeras mejoras pasan por dos vías: 

\begin{description}
   \item[Añadir funciones al control de luminarias] Cómo por ejemplo, un modo de simulación de presencia. Para, cuando estemos ausentes por varios días,  intentar evitar robos.
   \item[Mejorar la configuración] Para ello, por ejemplo, se podría generar de forma automática el script de configuración de los nodos. 
   \end{description} 
   
   Más allá de estas primeras modificaciones, el límite solo lo pone la imaginación. Con la tecnología actual se podrían añadir módulos para la apertura automática de puertas, control automático de toldos y persianas en función de la luz ambiental, climatización de la vivienda, alarmas de gases, y un largo etcétera. 
   
   Asimismo, se podrían añadir más formas de interacción con el sistema: aplicaciones nativas para <<smartphones>> y tabletas, comunicación por mensajería instantánea, control por voz o gestos. Una vez más el límite lo pone la creatividad.
   
   

\cleardoublepage
\addcontentsline{toc}{chapter}{Bibliografía}
\nocite{*}
\printbibliography
